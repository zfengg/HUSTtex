%!TEX program = xelatex
% HUSTtrans.tex @ https://github.com/zfengg/HUSTtex/tree/master/HUSTtrans
%
% 	A simple tex template for the translation task for undergraduate thesis at HUST.
%
% author: Zhou Feng @ 2019
%
% requirements:
%	TeX environments: TeXlive/MacTeX or MiKTeX
%	compiler: XeLaTeX
%
% ---------------------------------------------------------------------------- %
%                                   preamble                                   %
% ---------------------------------------------------------------------------- %
\documentclass[11pt,a4paper]{article}
%10.5pt equals 5 hao font. 

%----------------------------------------------------------------%
%% For the layout of paper
%\usepackage[tmargin=1in,bmargin=1in,lmargin=1.25in,rmargin=1.25in]{geometry}
\usepackage[top=2.54cm, bottom=2.54cm, left=3.18cm, right=3.18cm]{geometry}
\geometry{headsep=1em,footskip=2em}
\geometry{headheight=14pt}

%----------------------------------------------------------------%
%% For Math Symbols, 载入常用的数学包, 符号包
\usepackage{amsmath}
\usepackage{amsfonts}
\usepackage{amssymb}
\usepackage{mathrsfs}

%----------------------------------------------------------------%
%% For the linespace 行间距,段间距等等
\usepackage{setspace} 
% \usepackage{indentfirst} % then the first line of each title should start with a indent.
%定义标题和段落样式
%定义1.5倍行距
\renewcommand{\baselinestretch}{1.62}
\setlength{\baselineskip}{12pt} % this is used to set the fixed value of the lineskip
\setlength\parskip{\baselineskip} % set the space between the paragraphs, set the variable \parskip \baselineskip
% parindent
\setlength{\parindent}{0pt}

%----------------------------------------------------------------%
%% For the fonts (style, color, size).字体的大小,颜色,以及定义常用的字号
\usepackage{ctex}% If you are lazy, the CTEX suit is enough.
% Chinese Font
\usepackage{xeCJK}% For the Chinese through XeLaTex
\setCJKmainfont{SimSun} % set the mainfont of Chinese as songti. (serif) for  
\setCJKsansfont{SimSun} % sans serif font for \textsf
\setCJKmonofont{SimSun} % monospace font for \texttt
% \punctstyle{kaiming}  % Remove the space used by symbols like comma.
\setCJKfamilyfont{song}{SimSun}                             %宋体 song
\newcommand{\song}{\CJKfamily{song}}                        
\setCJKfamilyfont{kai}{KaiTi}                        		 %楷体2312  kai
\newcommand{\kai}{\CJKfamily{kai}}  
\setCJKfamilyfont{hwzs}{STZhongsong}                        %华文中宋  hwzs
\newcommand{\hwzs}{\CJKfamily{hwzs}}
% English Font
\usepackage{fontspec}% Then you can use the fonts installed at your device. 
\setmainfont{Times New Roman}
\setsansfont{Times New Roman}
\setmonofont{Times New Roman}

%\setsansfont{[foo.ttf]} % for the fonts at this default path.

% font colors: 利用definecolor自己可以定义颜色
\usepackage{xcolor}
\definecolor{MSBlue}{rgb}{.204,.353,.541}
\definecolor{MSLightBlue}{rgb}{.31,.506,.741}

% font sizes: (I use pinyin represents the corresponding size in Microsorft Word)
% \newcommand{\chuhao}{\fontsize{42pt}{\baselineskip}\selectfont}
% \newcommand{\xiaochuhao}{\fontsize{36pt}{\baselineskip}\selectfont}
% \newcommand{\yihao}{\fontsize{28pt}{\baselineskip}\selectfont}
% \newcommand{\erhao}{\fontsize{21pt}{\baselineskip}\selectfont}
% \newcommand{\xiaoerhao}{\fontsize{18pt}{\baselineskip}\selectfont}
% \newcommand{\sanhao}{\fontsize{15.75pt}{\baselineskip}\selectfont}
\newcommand{\sihao}{\fontsize{14pt}{18pt}\selectfont}
\newcommand{\xiaosihao}{\fontsize{12pt}{18pt}\selectfont}
\newcommand{\wuhao}{\fontsize{10.5pt}{18pt}\selectfont}
% \newcommand{\xiaowuhao}{\fontsize{9pt}{\baselineskip}\selectfont}
% \newcommand{\liuhao}{\fontsize{7.875pt}{\baselineskip}\selectfont}
% \newcommand{\qihao}{\fontsize{5.25pt}{\baselineskip}\selectfont}



%----------------------------------------------------------------%
%% For the header and footer. 页眉,页脚
 \usepackage{fancyhdr} % Then you can specialize the header and footer for your own use.
 %设置页眉样式
 \newcommand{\headstyle}{
 	\fancyhead[C]{ \hwzs\wuhao 华中科技大学本科生毕业设计(论文)参考文献译文}
 }
 %设置页脚样式
 \newcommand{\footstyle}{\fancyfoot[C]{\normalfont \thepage}
 	\fancyfoot[L]{\rule[5pt]{6.7cm}{0.4pt}}
 	\fancyfoot[R]{\rule[5pt]{6.7cm}{0.4pt}}
 }
 \pagestyle{fancy}
 \fancyhf{} %清空原有样式
 \headstyle
 \footstyle
 %定义一种新的格式叫做main
 \fancypagestyle{main}{%
 	\fancyhf{} %清空原有样式
 	\headstyle
 	\footstyle
 }
% \renewcommand{\headrulewidth}{0.4pt}
% \renewcommand{\footrulewidth}{0.4pt}
% %\renewcommand{\footrule}{\rule{\textwidth}{0.4pt}}
% \lhead{} \chead{华中科技大学本科生毕业设计(论文)参考文献译文} \rhead{}
% \lfoot{} \cfoot{\thepage} \lfoot{}


 

%----------------------------------------------------------------%
%% For the styles of sections at all levels
 %设置各个标题样式
 %不需要使用part和chapter层级
 \usepackage{titlesec}
 \usepackage{titletoc}
 \titleformat{\section}{\wuhao\bfseries}{\thesection.}{1em}{} %在section标题编号后面加个点
% \titleformat*{\section}{\wuhao\bfseries} % 设置标签的形式,5号加粗
 \titleformat*{\subsection}{\wuhao\bfseries}
 \titleformat*{\subsubsection}{\wuhao\bfseries}
 % 用titlespacing修改section与正文第一行之间的距离
% \titlespacing*{\section}{0pt}{}{}
% \titlespacing*{\subsection}{0pt}{}{}
% \titlespacing*{\subsubsection}{0pt}{}{}
 \newcommand{\sectionbreak}{\clearpage} %小节从新的一页开始
%根据学校要求设置新的section, subsection, subsection, subsubsection 以及 paragraph

% For the content of section and so on
\newcommand\seccontent{
	\wuhao %默认五号字体, 行间距为1.5*\baselineskip
    \setlength{\parindent}{2em} %首段缩进两个M字符
    \setlength{\parskip}{0pt}
    }


%----------------------------------------------------------------%
%% For the style of theorems, definitions, proofs and remarks 定义数学里面一些常用的环境
\usepackage{amsthm}
\newtheorem{thm}{\textbf{定理}}[section]
 %The section in [] can be replaced by chapter or subsection
 \theoremstyle{definition} \newtheorem{law}[thm]{Law}
 \theoremstyle{plain} \newtheorem{jury}[thm]{Jury}
 \theoremstyle{remark} \newtheorem*{marg}{Margaret}


%----------------------------------------------------------------%
%% For the caption and reference 图表及公式的编号规范
\usepackage{caption}
\captionsetup[figure]{labelformat=default, labelsep=quad,name={图}}
\captionsetup[table]{labelformat=default,labelsep=quad,name={表}}
%设置图表标题的计数方式
\renewcommand{\thefigure}{\thesection--\arabic{figure}} % set caption label style to 2-1 
\renewcommand{\thetable}{\thesection--\arabic{table}} % set caption label style to 2-1 
\captionsetup[figure]{labelfont=normalfont,textfont=normalfont} 
\captionsetup[table]{labelfont=normalfont,textfont=normalfont} 
%设置图表的autoref的格式
\newcommand{\reffig}[1]{图 \ref{#1}}
\newcommand{\reftab}[1]{表 \ref{#1}}
%公式的编号格式
\numberwithin{equation}{section}
\renewcommand\theequation{\arabic{section}--\arabic{equation}}


%\DeclareCaptionFont{hust}{\normalsize}
%\captionsetup{labelsep=quad}
%\captionsetup{font={hust,singlespacing}}
%\captionsetup[figure]{position=bottom}
%\captionsetup[table]{position=top}
%\setlength{\textfloatsep}{6pt}
%\setlength{\floatsep}{0pt}
%\setlength{\intextsep}{6pt}
\setlength{\abovecaptionskip}{15pt}
%\setlength{\belowcaptionskip}{0pt}

%重新定制figure和table环境使其更好使用(这样做好处在于方便,不用再打\centering, \label之类的,但是texstudio的autoref系统无法提前预知reference的名字,觉得合适的朋友可以newenvironment.)
%\newenvironment{generalfig}[3][htbp]{
%	\def \figcaption {#2}
%	\def \figlabel {#3}
%	\begin{figure}[#1]
%		\centering
%	}{
%		\caption{\figcaption} \label{\figlabel}
%	\end{figure}
%}
%\newenvironment{generaltab}[3][htbp]{
%	\def \tabcaption {#2}
%	\def \tablabel {#3}
%	\begin{table}[#1]
%		\caption{\tabcaption} \label{\tablabel}
%		\zihao{5}
%		\centering
%	}{
%	\end{table}
%}
%% For the figures and tabulars
\usepackage{graphicx} % To include graphixs
\usepackage{booktabs} % To create three line table including the commands toprule, bottomrule, and midrule
%\usepackage{colortbl} % 

%----------------------------------------------------------------%
%% For the tableofcontents, listoftables and listoffigures, 目录 
%参考文献翻译不需要管,之后制作论文tex文档的时候需要设定
\usepackage{tocloft}
\renewcommand\contentsname{目录}
\renewcommand\listfigurename{插图目录}
\renewcommand\listtablename{表格目录} 
%\titlecontents{section} [3cm] {\bf \large}{\contentslabel{2.5em}}{}{\titlerule*[0.5pc]{$\cdot$}\contentspage\hspace*{3cm}}
%\titlecontents{标题名}[左间距]{标题格式}{标题标志}{无序号标题}{指引线与页码}[下间距]

%----------------------------------------------------------------%
%% For the bibiliograph or reference and citation
\usepackage{natbib}
\renewcommand{\refname}{\wuhao\textbf{参考文献}}
\bibsep=0pt % 用来设置每个\bibitem之间的间距
%\renewcommand{\bibname}{参考文献} % For the document class 参考文献
%\newcommand{\upcite}[1]{\textsuperscript{\textsuperscript{\cite{#1}}}} % If you want the citation label to show at the uperscript position.

%----------------------------------------------------------------%
%\usepackage{makeindex} For the index 索引
\usepackage{listings} %For the code. 代码
%----------------------------------------------------------------%
%% For the hyperlink and bookmark 超链接及书签,这样生成的pdf中的引用直接点击链接即可到达目的地
\usepackage[bookmarks=true,colorlinks,linkcolor=black,citecolor=black,urlcolor=purple]{hyperref}% 设置超链接并修改风格
%----------------------------------------------------------------%
%% For the appendix, 附录

%----------------------------------------------------------------%
% For the titlepage 标题页,此处可以省略,建议直接使用官方给出的标题页即可
\usepackage{titling} 
\title{华中科技大学本科生毕业设计 \break{(论文)参考文献翻译} \vspace{-200em}}
\author{冯洲}
\date{}

%\makeatletter % change default title style
%\renewcommand*\maketitle{%
%	\begin{center}% 居中标题
%		\normalfont % 默认粗体
%		{\sihao \@title \par} % LARGE字号
%		\vskip -1000pt% %%%  标题下面只有1em的缩进或margin
%		{\global\let\author\@empty}%
%		{\global\let\date\@empty}%
%		\thispagestyle{fancy} %  不设置页面样式
%	\end{center}%
%	\setcounter{footnote}{0}%
%}
%\makeatother

%\pretitle{\vspace{-10ex} \begin{center}\sihao} 
%  \posttitle{\par\end{center}\vspace{-8mm}}
%\preauthor{} 
%  \postauthor{} 
%\predate{} 
%	\postdate{\vspace{-400pt}}


\begin{document}
%\maketitle

\section*{\centering  {\xiaosihao 中文} {\wuhao Abc123} \\ \centering \textcolor{red}{(宋体小4号, 字母、阿拉伯数字为 Times New Roman 5号加粗, 居中)}}
\vskip -1em
\section{前言}
\seccontent \textcolor{red}{(宋体5号, 字母、阿拉伯数字为 Times New Roman 5号加粗)}

Leslie  Lamport \textbf{Leslie  Lamport}开发的\textbf{\LaTeX}是当今世界上最流行和使用最为广泛的TeX宏集。它构筑在\textbf{Plain \TeX}的基础之上,并加进了很多的功能以使得使用者可以更为方便的利用\textbf{\TeX}的强大功能。使用\textbf{\LaTeX}基本上不需要使用者自己设计命令和宏等,因为\textbf{\LaTeX}已经替你做好了。因此,即使使用者并不是很了解\textbf{\TeX},也可以在短短的时间内生成高质量的文档。对于生成复杂的数学公式,\textbf{\LaTeX}表现的更为出色\textbf{x}。\textbf{ \LaTeX}自从八十年代初问世以来,也在不断的发展.最初的正式版本为\textbf{2.09},在经过几年的发展之后,许多新的功能,机制被引入到\textbf{\LaTeX}中。在享受这些新功能带来的便利的同时,它所伴随的副作用也开始显现,这就是不兼容性。

\song
标准的\textbf{\LaTeX 2.09} 引入了“新字体选择框架”(\textbf{NFSS})的\textbf{\LaTeX、SLiTEX,AMS-\LaTeX}等等,相互之间并不兼容.这给使用者和维护者都带来很大的麻烦。为结束这种糟糕的状况,\textbf{FrankMittelbach} 等人成立了\textbf{ATeX3}项目小组,目标是建立一个最优的,有效的,统一的,标准的命令集合。即得到\textbf{\LaTeX}的一个新版本\textbf{3}.这是一个长期目标,向这个目标迈出第一步就是在\textbf{1994}年发布的\textbf{\LaTeXe}



\textcolor{red}{(宋体5号, 行间距固定1.5倍行距,字符间距为标准)}

\subsection{字体风格} \seccontent
\noindent 我是\LaTeX{}中的11pt大小的宋体. 我
{\wuhao 我是五号宋体}\textbf{3156}416 AaBc
\subsection{字体颜色} \seccontent
\textcolor{MSBlue}{微软蓝色}
\subsubsection{第三级标题}



\section{方程及图表}
$ \times\times\times\times\times\times\times\times\times\times\times\times\times\times\times\times\times\times\times\times $,其 $ \times\times\times\times\times$可表示如下:
\begin{equation}
	E_{1}=A_{1}sin\!\left(2\pi f_{1}t+\varphi_{01}+\varphi_{path1} \right)
\end{equation}
\begin{equation}
	E_{2}=A_{2}sin\!\left(2\pi f_{2}t+\varphi_{02}+\varphi_{path2} \right)
\end{equation}

$ \times\times\times\times\times\times\times\times\times\times\times\times\times\times\times\times\times\times\times\times $  (如\reftab{table1}所示)

\begin{table}[htpb]
	\centering
	\caption{样表}
	\label{table1}
	\begin{tabular}{cccc}
		\toprule
		$ \times\times\times\times\times $ & $ \times\times\times $ & $ \times\times\times $ & $ \times\times\times $ \\
		\hline
		$ \times\times\times\times\times $ & $ \times\times $       & $ \times\times $       & $ \times\times $       \\
		$ \times\times\times\times\times $ & $ \times\times $       & $ \times\times $       & $ \times\times $       \\
		$ \times\times\times\times\times $ & $ \times\times $       & $ \times\times $       & $ \times\times $       \\ 	    	\cline{2-4}
		$ \times\times\times\times\times $ & $ \times\times $       & $ \times\times $       & $ \times\times $       \\
		\bottomrule
	\end{tabular}
\end{table}
\textcolor{red}{(表标题:位于表格上方,宋体5号,字母、阿拉伯数字为Time New Roman 5号,表内容:宋体5号,字母、阿拉伯数字为Time New Roman 5号)\\ ``\fbox{\phantom{a}}''表示空格}

$ \times\times\times\times\times\times\times\times\times\times\times\times\times\times\times\times\times\times\times\times $  (如\reffig{testfig}所示)
\begin{figure}[htbp]
	\centering
	\includegraphics[width=\textwidth]{testmindmap}
	\caption{测试图片, 因为学校模板给的word中的图片就是从这上面截取的部分,所以另存为PNG之后就是这个样子}
	\label{testfig}
\end{figure}

$ \times\times\times\times\times\times\times\times\times\times\times\times\times\times\times\times\times\times\times\times $  (如\reffig{E8}所示)

\begin{figure}[htbp]
	\centering
	\includegraphics[width=0.5\textwidth]{E8Petrie}
	\caption{测试图片: E8 李群}
	\label{E8}
\end{figure}

\textcolor{red}{(图标题:位于图下方,宋体5号,字母、阿拉伯数字为Time New Roman 5号)}


\section{列举环境} \seccontent
\begin{description}
	\seccontent
	\item[列表]
	\item[枚举] \begin{enumerate}
			\item 图
			\item 表
		\end{enumerate}
	\item[列举] \begin{itemize}
			\item hello,world!
			\item 你好!
		\end{itemize}
\end{description}


\sectionbreak
\section{为什么写这个tex文件}\seccontent
\subsection{背景} 每年大家都在抱怨数学公式难敲,而且敲出来不好看!确实,word的宏, 如MathType, AxMath等, 可以通过点点点来勉强解决问题,但是生成出来的公式都是一块块的,间距和大小都不好控制,更别谈交叉引用了,想起来就烦!(可能文献翻译不会有交叉引用的烦恼)。 在不考虑学习成本的情况下,\LaTeX\ 可以轻松搞定特别复杂的公式,而且排版特别漂亮。跟Word所见即所得(WYSIWYG: what you see is what you get)不同,\LaTeX 是所想即所得(WYWIWYG: what you want is what you get), 利用代码告诉计算机你想要什么就行了。想必大家在做文档的时候,肯定有类似的烦恼:欸? 我这里怎么忘了啥啥啥了, 那里怎么看起来有点不对劲? 这些对于\LaTeX 来说都不是问题,你只需关注内容就够了,深受科研工作者的喜爱。更何况现在基本上每个学科的主流期刊论文都是 \LaTeX 编写,而且基本都提供\LaTeX 模板。其实很多国外大学的毕业论文都提供官方的\LaTeX 模板的,可怜的我们还要自己敲非官方的 :(

作为一个优秀的排版软件,\LaTeX 远不止敲公式这点儿实力,除了刚刚提到的方便的交叉引用,优秀的参考文献管理BiB\TeX(\url{http://www.bibtex.org/}),各种各样的格式控制,甚至可以加入编程语句来控制等等\ldots\ 还有,CV, 信件,学术展示,报告等都可以通过\LaTeX 来变得更加优美~(反正我上次去的数学联会上的幻灯片都是\LaTeX\ Beamer做的,没人用PPT!) \LaTeX 写报告的体验极佳!
哎\~{}说到这儿,我更加嫌弃Word了哈哈\~{}

\subsection{前期} 起初, 大概几个月前,我在网上搜现成的华科\LaTeX 参考文献翻译模板,我还真找到了一个02级学长发布在github上的套装(\url{https://hust-latex.github.io/}),然后fork到了自己的账户里面。然而就在前天,我准备用的时候(我们课题组说可以拖到下学期hhh, 而且前段时间也挺忙的。),我方了。不仅它的格式要求和现在完全不一样了,而且这个模板文件搞得特别麻烦。麻烦在它给的是ins和dtx文件(理解成生成sty和cls文件的前期材料就可以啦),编译他们的时候,还要 \verb|l3docstrip.tex文件| (\LaTeX3 里面用来分离ins中文本的tex文件)。好,这些都搞定之后,我得到了最终的tex文件,编译出了pdf,但是不仅缺字体,而且格式不符合要求。于是,我开始着手直接敲tex头文件,而且对前提条件的要求不能太高,要简单粗暴,要最后直接修改样本就可以使用。最后就是这个tex文件了。

后来,我也找到了一个17年的学长的研究生毕业论文 cls 文件(\url{https://github.com/skinaze/HUSTPaperTemp}),但是也是不好编译,容易出问题,而且cls文件的修改的语法要比tex文件麻烦,更何况这个的其他格式不符合我科论文翻译的要求。

\subsection{优缺点}
\begin{description}
	\seccontent
	\item[优点1] Tex文件简单粗暴,想要修改什么直接在头文件里面改就行了, 不需要再编译ins或者dtx之类的。而且我的注释写得还比较详细的,参数的含义也好懂。根据不同要求可以再进行改进,完全定制化。
	\item[优点2] 编译的必备条件不高,只要正确配置了 TexStudio 应该都可以正常编译,记得编译器选 Xe\LaTeX。
	\item[优点3] 学习成本低,直接找到样本pdf相应位置修改对应的代码就可以的。
	\item[缺点] 对于第一使用\LaTeX 的朋友来说,可能花一段时间熟悉适应。tex文件头可能显得比较臃肿,第一次编译起来大概得 2 秒左右,后面就 0.5 秒吧。不过再TexStudio的快捷键的帮助下用起来挺不错的。
\end{description}


\section{使用说明}\seccontent
\begin{description}
	\seccontent
	\item[基本信息] 作者:冯洲 \href{https://github.com/zfengg}{@zfengg}. 版本信息:2019/1/23, v1.0发布在 \href{https://github.com/zfengg/HUSTtex}{HUSTtex} 仓库。 如果有任何建议及纠正,欢迎 \href{https://github.com/zfengg/HUSTtex/issues}{Issue} 或者 \href{https://github.com/zfengg/HUSTtex/pulls}{Pull request}!
	
	\item[必备条件]  安装最新版本的 \href{http://www.tug.org/texlive/}{TeXLive}(推荐)或 \href{http://miktex.org/}{MiKTeX}。请确保所有宏包都更新至最新。因为中文支持利用的是包\textbf{XeCJK},所以编译器请使用 Xe\LaTeX。 编辑器推荐 \href{http://texstudio.sourceforge.net/}{TeXstudio}。 
	此文件在 Windows, Linux, MacOS 编译通过。
	
	\item[章节内部及其他环境内部的格式] 请在每个环境或章节后添加 $\backslash$seccontent (5号宋体,1.5倍行距)。 例如 
	 \verb|\section\seccontent|。
	如正文中的数字和字母要加粗,Ctrl+B即可。
	
	\item[图表引用] 图标的编号及题注已设计符合要求,如要引用, 请使用$\backslash$reffig$\lbrace\rbrace$ 引用图,$\backslash$reftab$\lbrace\rbrace$引用表格,以达到要求样式。
	
	\item[公式交叉引用] 方程的编号已调好,但是引用的格式我没有另外设计,因为引用的地方可能把公式叫法不同,引用请使用自带的$\backslash$ref$\lbrace\rbrace$。
	
	\item[距离控制] 这个tex文件的距离控制可能不太精细,如果有具体的标准数值请联系,作者来完善!
	
	\item[页眉页脚] 页眉页脚的样式已经调好,距页边缘的应该也没错。如果知道精确的距离请\href{https://github.com/zfengg/HUSTtex/issues}{Issue},我马上调整,谢谢!
	\item[超链接及书签] 利用hyperref包,每个link, cite, url已调整成超链接,点击即可到达相应位置。PDF书签及链接的样式请在头文件处根据自己的喜好修改。
	
	\item[参考文献] 其实论文翻译这块给的模板并没有要求参考文献,可自行删除,但是这里提供了两种参考文献的样式:第一种使用\href{http://www.bibtex.org/}{BiB\TeX}\ 引用 \verb|.bib| 文件。
	
	第二种直接利用环境 \verb*|\thebibliography| 小技巧:可以把使用BiB\TeX 生成的 \verb*|.bbl| 文件内容粘到 \verb*|.tex| 文件中以达到文件的独立性。
	
	\item[其他] 如脚标,目录,图表目录之类的,参考文献翻译没有要求,所以我就没设计,应该不会用。如果有什么特别的需要,就直接修改头文件相应内容即可。
	
	\item[注] 这个文献翻译只包含正文部分,学校给的封面按照要求打印即可。还有参考文献原文,一般收到的是 PDF,在生成 PDF 之后,利用处理PDF的软件组合一下就好啦!
\end{description}

\appendix
\phantom{\cite{eXGuru1999,DonaldKnuth1984,Rudin}}

% ---------------------------------------------------------------------------- %
%                                   reference                                  %
% ---------------------------------------------------------------------------- %
\bibliographystyle{plain}\label{bibtexref}
\bibliography{HUSTtrans}
\begin{thebibliography}{1}
	\label{latexref}
	\seccontent
	\bibitem{rudin} 王静康,张凤宝,夏淑倩等.论化工本科专业国际认证与国内认证的“实质性”.高等工程教育研究,2014,5:1-4

	\bibitem{stone} Stone J A, Howard L P. A simple technique for observing periodic nonlinearities in Michelson interferometers. Precision Engineering,1998,22(4):220-232

\end{thebibliography}
%	
\end{document}
